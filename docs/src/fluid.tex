\documentclass{article}
\usepackage{amssymb}
\usepackage{amsmath}
\usepackage{graphicx}
\usepackage{comment}

\begin{document}

\title{Basic Fluid Simulation}
\author{Christian Oeien}
\date{August 19, 2016}
\maketitle

\begin{abstract}

Implementation of a 2D fluid simulator is presented.
A fine-resolution picture flows on the fluid and
the viscosity and density of the fluid at a position
is mapped from the color at the moving picture.
\end{abstract}

\section{Introduction}

Computer simulation of continuous phenomena in general
has two main problems.  Firstly a mathematical model must
be provided that predicts the phenomenon.  This may be
a subset of the physical laws involved.
Secondly an implementation must evaluate a finite set
of discrete calculations that renders a simulation.
Care must be taken to prevent that numeric rounding
nullifies an expected result,
or that instabilities arise from the periodicity
introduced by taking finite steps.

When moving the picture along the fluid, I use randomized
sub-pixel position offset, a technique generally referred to
as the Monte-Carlo method.  The mentioned instability
may be reduced by using the Leap Frog technique.
\begin{comment}
described in \cite{leapfrog}.
\end{comment}

The Navier-Stokes equations describes the mechanics
of a fluid in general.  My simulation considers
a Newtonian fluid.
Principles modeled by my simulator are
the pressure-velocity relation, advection,
venturi-effect and rotation.  Calculations are done
in synchronous steps on a grid of cells.

\section{Material and Methods}

By using generic programming, as the C++ template system
provides, I factored out the cell-system.  This was
specialized in an implementation of a wave-simulation as
well as the fluid simulator as reported here.

The generic framework takes a cell-type implementation as
parameter.  The cell-type contains the physical variables
and interactions its with four neighboring cells.

\section{Results}

The implementation is found at
\texttt{HTTP://github.com/biotty/rmg/graphics/flow/fluid.cpp}
and the generic header-files in the same directory.

Each cell in the flow-grid is responsible for mathematically
modelling the physics of the flow, in this case the formulae
and state for the physics of a fluid at a certain
viscosity and density.  The grid itself is part of the
generic program and is agnostic to any physics.

I found that the principle of advection is critical for
modeling a fluid; without advection the movements in the flow
are not fluid-like.  The venturi-effect is then needed to
make any turbulence happen as naturally in a fluid.
The correctness of the movement, especially when turbulent,
is further improved by modelling rotation.

The rendering of a picture on the fluid is done at a
different resolution than the fluids cell-grid and the
movement is interpolated with a sub-cell random offset.
This prevents some pixels to be smeared out due to
pixel-rounding.

The fluid-parameter feedback is based on color in a cell,
and in this case we need to take the most common color
in the cell.  The picture is color-mapped with 256
color entries.  I initially quantize the input-picture
using an external quantizer tool.

Input forces at given locations are used to accelerate
the fluid in order to create movement.
Viscosity and density for certain colors are also
provided as input, as well as default values.

\section{Discussion}

It is not difficult to trigger instability in this
simulator, by providing big values for forces.
But if velocities are given instead of forces,
instability occurs more frequently.
Care must be taken with any fluid model,
but the 'leap frog' technique would reduce
likelihood of this problem occurring.
I do not implement that method, and instead take
synchronous steps of the physical parameters in the cells.
In this way, the physics being modeled by the cells are clear
to see in the implementation, and I believe it would be
obscured by the more stabilizing numerical technique.
It is not difficult to provide inputs
that results in a stable simulation.

The fluid model and the generic cell-framework are both
fixed at 2D.  The physics are the same and would be
trivial to translate to 3D, but the implementation
would contain more math and structures compared to the
physics formulae.  Therefore 2D is more effective
to demonstrate the physical correctness.  Also, the
visual outputs are even easier to study and
appreciate in 2D.  The computation-time would also
increase significantly if done for 3D.

Non-Newtonian fluids change parameter with the amount of
stress or tension in the fluid.
My model is Newtonian, but note that the fluid-parameters
are parametric per location, and are subject to change while
the fluid moves.

\begin{comment}
\bibliographystyle{plain}
\bibliography{fluid}
\end{comment}

\end{document}
